%%%% ijcai19-multiauthor.tex

\typeout{IJCAI-19 Multiple authors example}

% These are the instructions for authors for IJCAI-19.

\documentclass{article}
\pdfpagewidth=8.5in
\pdfpageheight=11in
% The file ijcai19.sty is NOT the same than previous years'
\usepackage{ijcai19}

% Use the postscript times font!
\usepackage{times}
\usepackage{soul}
\usepackage{url}
\usepackage[hidelinks]{hyperref}
\usepackage[utf8]{inputenc}
\usepackage[small]{caption}
\usepackage{graphicx}
\usepackage{amsmath}
\usepackage{booktabs}
\urlstyle{same}

% the following package is optional:
%\usepackage{latexsym} 

% Following comment is from ijcai97-submit.tex:
% The preparation of these files was supported by Schlumberger Palo Alto
% Research, AT\&T Bell Laboratories, and Morgan Kaufmann Publishers.
% Shirley Jowell, of Morgan Kaufmann Publishers, and Peter F.
% Patel-Schneider, of AT\&T Bell Laboratories collaborated on their
% preparation.

% These instructions can be modified and used in other conferences as long
% as credit to the authors and supporting agencies is retained, this notice
% is not changed, and further modification or reuse is not restricted.
% Neither Shirley Jowell nor Peter F. Patel-Schneider can be listed as
% contacts for providing assistance without their prior permission.

% To use for other conferences, change references to files and the
% conference appropriate and use other authors, contacts, publishers, and
% organizations.
% Also change the deadline and address for returning papers and the length and
% page charge instructions.
% Put where the files are available in the appropriate places.

\title{Graph-based Transfer Techniques in Coordination Graph RL}

\author{ % what will be our order?
First Author\footnote{Contact Author}\and
Second Author
\affiliations
Case Western Reserve University\\
\emails  % order?
\{first, second\}@case.edu
}

\begin{document}

\maketitle

\begin{abstract}
ABSTRACT HERE
\end{abstract}

\section{Introduction} % (fold)
\label{sec:introduction}

Introduction here.

% section introduction (end)

\section{Background and Related Work} % (fold)
\label{sec:background_and_related_work}

...MARL in general...

...Coordination Graph Learning by Guestrin et al....

...Transfer Methods (Julie Kaplan)...

% section background_and_related_work (end)

\section{Deletion Transfer} % (fold)
\label{sec:deletion_transfer}

...the 4 existing basis transfer methods...

% (*new) ...delete and connect-map instead of clique...

...proxy deletion via monte carlo expected reward after deletion...

(we should consider how to train on original policies with the modified policies)

(we should consider the influence of the policy representation on how this works)

% section deletion_transfer (end)

\section{Addition Transfer} % (fold)
\label{sec:addition_transfer}

...problem of where to add in the graph...

...no supergraphs / exponential supergraphs...

(*) ...using the expected reward after deletion as heuristic for load...

(*) ...partition and mirror / partition and copy...

...heuristic-free methods (e.g. L1 norm of policy weights / mean activation)

% section addition_transfer (end)

\section{Experimental Results} % (fold)
\label{sec:experimental_results}

...domain 1 and 2...

...graphs...

...visualizations and charts etc go here...

% section experimental_results (end)

\section{Conclusion} % (fold)
\label{sec:conclusion}

% section conclusion (end)

\end{document}

